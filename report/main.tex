\documentclass[a4paper]{article}

\usepackage[utf8]{inputenc}
\usepackage[russian]{babel}

\usepackage{mathtext}
\usepackage[T2A]{fontenc}

\usepackage{listings}
\usepackage{color}


\usepackage{algorithmicx}
\usepackage{algpseudocode}

\usepackage{amssymb}
\usepackage{amsopn}
\usepackage{mathtools}

\begin{document}

\begin{titlepage}

	\begin{center}

		\large Федеральное государственное автономное образовательное учреждение высшего образования \\
		\large «Санкт-Петербургский политехнический университет Петра Великого» \\
		\large Институт компьютерных наук и технологий \\
		\large Кафедра «Компьютерные интеллектуальные технологии» \\[6cm]

		\huge {\bf Курсовая работа} \\[0.5cm]
		\large {\bf Информационная система автомобилестроительного предприятия} \\[0.1cm]
		\large по дисциплине «Базы данных» \\[6cm]

	\end{center}

    \begin{center}
        \begin{minipage}[t]{4cm}
            \begin{flushleft}
                Выполнил
                студент гр. 23506/1
            \end{flushleft}
        \end{minipage}
        \hfill
        \begin{minipage}[t]{4cm}
            \begin{flushright}
            О.Д. Романов
            \end{flushright}
        \end{minipage} \\[0.5cm]

        \begin{minipage}[t]{4cm}
            \begin{flushleft}
                Руководитель
                старший преподаватель
            \end{flushleft}
            \flushleft
        \end{minipage}
        \hfill
        \begin{minipage}[t]{4cm}
            \begin{flushright}
                Н.В. Андреева
            \end{flushright}
        \end{minipage}
    \end{center}

    \begin{flushright}
        $\ll$\underline{\hspace{0.5cm}}$\gg$ \underline{\hspace{1.5cm}} 2017
    \end{flushright}

	
	\vfill

	\begin{center}
	    \large Санкт-Петербург\\
	    \large \the\year
	\end{center}
 
\end{titlepage}

\tableofcontents
\newpage

\section{Исходные данные}

\subsection{Техническое задание}
Структурно предприятие состоит из цехов, которые в свою очередь подразделяются на участки.

Категории изделий, выпускаемых предприятием: грузовые, легковые автомобили, автобусы, сельскохозяйственные, дорожно-строительные машины, мотоциклы и прочие изделия.
Каждая категория изделий имеет специфические, присущие только ей атрибуты.
Например, для автобусов это вместимость, для сельскохозяйственных и дорожно-строительных машин - производительность и т.д.

По каждой категории изделий может собираться несколько видов изделий (под видом изделия понимается конкретная его разновидность / марка - например, автомобиль KIA Rio).
По конкретным экземплярам каждого вида ведётся журнал, где отмечаются даты завершения различных этапов жизненного цикла изделия: изготовление (сборка) / тестирование / передача дилеру / гарантийный ремонт.

Предприятие в основном состоит из производственных цехов, но также есть несколько вспомогательных (например, ремонтный, тестировочный).

Каждая категория изделий собирается в своём производственном цехе (в одном цехе может собираться несколько категорий изделий).
Цех структурно состоит из участков, на каждом из которых выполняется один вид работ: изготавливается определённая часть изделия (например, двигатель) либо производится сборка изделия в целом.
С каждой категорией изделия ассоциируется свой набор работ; другими словами, каждая категория в процессе изготовления должна пройти определённый набор участков в цехе.

Каждой категории инженерно-технического персонала (инженеры, технологи, техники) и рабочих (сборщики, токари, слесари, сварщики и пр.) также характерны атрибуты, свойственные только для этой группы.
Рабочие объединяются в бригады, которыми руководят бригадиры.
Бригадиры выбираются из числа рабочих.
Работу цеха возглавляет начальник цеха, а работу на участке - начальник участка, в подчинении которого находится несколько мастеров.
Каждый мастер координирует работу одной или нескольких бригад (но, в отличие от бригадира, не входит в состав конкретной бригады).
Мастера, начальники участков и цехов назначаются из числа инженерно-технического персонала.
Каждый начальник может руководить только одной структурной единицей (в т.ч. начальник одной структурной единицы не может быть в то же время начальником другой).

Работу по сборке конкретной категории изделия на определенном участке выполняет одна бригада рабочих, при этом она может обслуживать несколько участков / категорий и на одном участке может работать несколько бригад.

\subsection{Виды запросов в информационной системе}
\begin{enumerate}

    \item Перечень видов изделий по категории, собираемой указанным цехом.
    В последней строке вывести общее число собираемых видов изделий.

    \item Количество экземпляров изделий каждого вида каждой категории, собранных предприятием за определенный отрезок времени.
    В последней строке вывести общее число собранных изделий.
    Примерный вид результата:

    \begin{tabular}{|p{6cm}|p{2cm}|p{1cm}|} \hline
        Категория & Вид & Кол-во \\ \hline
        Автобусы & АКБ-12 & 8 \\ \hline
        Автобусы & АКБ-05 & 0 \\ \hline
        Автомобили & ИЖ-400 & 12 \\ \hline
        ... & ... & ... \\ \hline
        Всего собрано (12.05.2010-18.07.2010): & & 48 \\ \hline
    \end{tabular}

    \item Данные о кадровом составе (ФИО, должность) по указанным категориям инженерно-технического персонала и рабочих;
    \item Число и перечень участков предприятия и их начальников (с указанием цехов).
    \item Перечень работ, которые проходит указанный вид изделия.
    \item Состав бригад, работающих на указанном участке указанного цеха: ФИО рабочего, номер бригады, номер участка, номер цеха. Отсортировать по номеру бригады.
    \item Перечень мастеров (ФИО) указанного участка указанного цеха и номера бригад, работы которых они координируют.
    \item Информация о цехах, в которых в настоящий момент собирается больше видов изделий, чем в среднем приходится на каждый производственный цех предприятия: номер цеха, название цеха, кол-во собираемых видов изделий, среднее количество видов изделий по цехам предприятия.
    \item Состав бригад, участвующих в сборке указанной категории изделия.
    \item ФИО и должности работников цеха, в котором собирается больше всего категорий изделий.

\end{enumerate}

\section{Сущности, их харрактеристики и связи}
В ходе анализа начального описания предметной области были выявлены следующие сущности:

\begin{enumerate}

    \item

\end{enumerate}

\end{document}